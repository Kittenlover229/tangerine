\documentclass{article}
\pagestyle{empty}

\newcommand{\s}[1]{\texttt{#1}}
\newcommand{\In}{\paragraph{Input:}}
\newcommand{\Out}{\paragraph{Output:}}
\newcommand{\Note}{\paragraph{Note:}}
\newcommand{\Hint}{\paragraph{Hint:}}

\begin{document}

\section{Simple Stuff}
Prerequisites: control flow (branching, iteration), IO, arithmetic, atomic types.

\subsection{Equation of a Line $\star$}
\In
Two integers $k$ and $b$.
\Out
Such value $x$, that it satisfies the equation $kx+b=0$. 

\subsection{Wait, what? $\star\star$}
\In
Two integers $a$ and $b$.
\Out
The product of $a$ and $b$.
\Note
You may not use the multiplication operation.

\subsection{Late'o'clock $\star\star$}
\In An integer $0 \leq h < 24$. Hours on a clock.
\Note Convert the given time $h$ to the 12-hour clock format.
\Out First the time $h$ in 12-hour clock format, then \s{"am"} or \s{"pm"}
depending on the time.

\subsection{Quadratic Equation $\star\star$}
\In
Three integers $a$, $b$ and $c$.
\Out
Find all values of $x$, such that $ax^2 + bx + c=0$. 
\Note
If there are no possible values of $x$ output \texttt{"NaN"} (not a number). 
The values should not be repeated.

\subsection{Qubic Equation $\star\star\star$}
\In
Four integers $a$, $b$, $c$ and $d$.
\Out
Find all values of $x$, such that $ax^3 + bx^2 + cx + d = 0$. 
\Note
If there are no possible values of $x$ output \texttt{"NaN"} (not a number). 
The values should not be repeated.
\Hint
use Cardano's formula.

\subsection{Minmaxed $\star$}
\In
Two integers, $a$ and $b$.
\Out
Two integers, first the largest of them two, next the smallest.

\subsection{Sigma for Sum $\star\star$}
\In
An integer $a$ such that $1 \leq a \leq 10^{10^{10}}$.
\Out
The sum all the integers $1 + 2 + \dots + a$.
\Hint
Loop isn't the only way to go.

\subsection{Factor!al $\star\star\star$}
\In
An integer $a$ such that $1 \leq b \leq 10^{5}$.
\Out
The product all the integers $1 \times 2 \times \cdots \times b$.
\Hint
Lookup the arguments for \s{range} in the official Python3.x documentation.

\subsection{Minmaxed 2: The Sequel $\star\star\star$}
\In
Two integers, $a$ and $b$.
\Out
Two integers, first the largest of them two, next the smallest.
\Note
You may  only use \s{min()} or \s{max()}, not both. You may not use branching.

\section{Turtle or Tortoise?}
Prerequisites: \s{turtle} module, the entire previous section.

\subsection{Fair Square $\star$}
\In
An integer $A$ such that $10 \leq A \leq 100$.
\Out
Using \s{from turtle import Turtle}'s methods like 
\s{forward} and \s{right} draw a square of length $A$.

\subsection{Fair Ngon $\star\star$}
\In
Two integers, $A$ such that $10 \leq A \leq 100$ and $N$
such that $2 \leq N \leq 20$.
\Out
Using \s{Turtle} draw a regular polygon (an $N$-gon) with $N$ sides and side
length $A$. Ensure that the turtle finishes in the same position as it started
in. The turtle shouldn't draw over itself at any point.
\Hint
Loops are your friend.

\subsection{Trigonometry BFF $\star\star\star$}
\In
Two integers, $a$ and $b$.
\Out
Using \s{Turtle} draw a graph of the function $y = a * sin(\frac{\pi x}{10}) + b$.
From $0$ to $20$ and a graph of the function $y = b$. Print the final position
of the turtle.
\Hint
You can get $\sin$ and $\pi$ with \s{from math import pi, sin}, they are accurate
enough for this purpose.

\subsection{The Fair Ngon $\star\star\star\star\star$}
\In
Two integers, $A$ such that $10 \leq A \leq 100$ and $N$
such that $2 \leq N \leq 20$.
\Out
Using \s{Turtle} draw a regular polygon (an $N$-gon) with $N$ sides and side
length $A$. Ensure that the turtle finishes in the same position as it started
in. You are only allowed to control the turtle with \s{penup}, \s{pendown},
\s{goto}.
\Hint
Trigonometry might help.

\end{document}
