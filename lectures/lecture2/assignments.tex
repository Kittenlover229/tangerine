\Note Use \s{inputs = list(map(int, input().split()))} to parse the list of numbers into a variable,
no need to know it works for now, just assume it's magic. This transforms an
input of \s{1 2 3 4 5 6} into \s{[1, 2, 3, 4, 5, 6]}.


\begin{task}{Fair Square}{1}
    \In
    An integer $A$ such that $10 \leq A \leq 100$.
    \Out
    Using \s{from turtle import Turtle}'s methods like 
    \s{forward} and \s{right} draw a square of length $A$.
\end{task}

\begin{task}{Fair Ngon}{2}
    \In
    Two integers, $A$ such that $10 \leq A \leq 100$ and $N$
    such that $2 \leq N \leq 20$.
    \Out
    Using \s{Turtle} draw a regular polygon (an $N$-gon) with $N$ sides and side
    length $5A$. Ensure that the turtle finishes in the same position as it started
    in. The turtle shouldn't draw over itself at any point.
    \Hint
    Loops are your friend.
\end{task}

\begin{task}{Trigonometry BFF}{3}
    \In
    Two integers, $a$ and $b$.
    \Out
    Using \s{Turtle} draw a graph of the function $y = a * sin(\frac{\pi x}{10}) + b$.
    From $0$ to $200$ and a graph of the function $y = b$. Print the final position
    of the turtle.
    \Hint
    You can get $\sin$ and $\pi$ with \s{from math import pi, sin}, they are accurate
    enough for this purpose.
\end{task}

\begin{task}{The Fair Ngon}{5}
    \In
    Two integers, $A$ such that $10 \leq A \leq 100$ and $N$
    such that $2 \leq N \leq 20$.
    \Out
    Using \s{Turtle} draw a regular polygon (an $N$-gon) with $N$ sides and side
    length $10A$. Ensure that the turtle finishes in the same position as it started
    in. You are only allowed to control the turtle with \s{goto}.
    \Hint
    Trigonometry might help.
\end{task}

\begin{task}{Tick Space Tick Space Tick}{2}
    \In
    Two integers, $10 \leq L \leq 100$ and $1 \leq N \leq 15$.
    \Out
    Draw a horizontal dotted line of $N$ segments. 
    The length of each segment should be $L$.
    The space between two segments should also be $L$.
    \Note
    The turtle should start and end the drawing with a filled segment.
    \Hint
    Make use of \s{turtle.penup}, \s{turle.penup} and \s{turtle.isdown}.
\end{task}

\begin{task}{Fib}{2}
    \In
    An integer $n$, $n > 1$.
    \Out
    All terms from 0th to $n$th (inclusive) of the Fibonacci sequence.
    The Fibonacci sequence is defined as follows.
    $$F_0 = 0$$
    $$F_1 = 1$$
    $$F_n = F_{n - 1} + F_{n - 2}$$
    \Note
    You can solve this with both loops and recursion, maybe try it both ways?
    You may use \href{https://oeis.org/}{this website} to check how correct your result is.
    \begin{ExampleIO}
    \egio{14}{0\\1\\1\\2\\3\\5\\8\\13\\21\\34\\55\\89\\144\\233\\377\\}
    \end{ExampleIO}
    \end{task}

    \begin{task}{Blaise's Blessing}{3}
    \In
    An integer $N$, $N > 0$.
    \Out
    $N$ rows of the pascal triangle.
    \Hint
    The air is stale, open the \textit{window} please.

    \begin{ExampleIO}
    \egio{4}{1\\1 1\\1 2 1\\1 3 3 1\\1 4 6 4 1}
    \end{ExampleIO}
\end{task}


\begin{task}{Average}{1}
    \In
    A list of space-separated numbers $a_i$ of length $n \geq 1$. 
    \Out
    An average of all numbers, $\frac{a_1 + a_2 + \dots + a_n}{n}$.

    \begin{ExampleIO}
    \egio{1 2 3 4 5 6}{3.5}
    \egio{0}{0}
    \egio{1 -1}{0}
    \end{ExampleIO}
\end{task}


\begin{task}{Furthest Apart}{2}
    \In
    A list of space-separated numbers $a_i$ of length $n \geq 2$. 
    \Out
    The largest distance between two numbers.
    \Note 
    Say in the 1st example the most spread-apart numbers are 1 and 5, and the
    distance between them is 4, the answer. In the second all the numbers are the
    same, the distance between any number and itself is 0.
    \begin{ExampleIO}
    \egio{1 2 3 4 5}{4}
    \egio{3 3 3 3 3}{0}
    \end{ExampleIO}
\end{task}


\begin{task}{\reflectbox{Reversed}}{2}
    \In
    A list of space-separated numbers $a_i$ of length $n \geq 0$.
    \Out
    The same list in reverse order.
    \Note 
    Try and reverse the list in-place, without creating a new one to copy the elements into.
    
    \begin{ExampleIO}
    \egio{1 4 9}{9 4 1}
    \egio{1}{1}
    \end{ExampleIO}
\end{task}

\begin{task}{ROT K}{3}
    \In
    First line is an integer $-2n \leq K \leq 2n$.
    The next line is a list of space-separated numbers $a_i$ of length $n \geq 0$.
    \Out
    The same list with all of its elements shifted by $K$, to the right if $K$ is 
    positive and to the left if negative. When $K = 0$ the list should stay intact.
    \Note 
    Try and do this in-place, without creating a copy of a list!
    
    \begin{ExampleIO}
    \egio{2\\1 2 3 4 5}{4 5 1 2 3}
    \egio{-1\\1 2 3 4 5}{2 3 4 5 1}
    \egio{-6\\1 2 3 4 5}{2 3 4 5 1}
    \egio{0\\1 2 3 4 5}{1 2 3 4 5}
    \end{ExampleIO}
\end{task}

\begin{task}{Shufflepuff}{1}
    \In
    A list of space-separated numbers $a_i$ of length $n \geq 1$.
    \Out
    The same list shuffled in any way you want. The output list must not be 
    identical to the input list. The shuffle of the same list must be the
    same across program runs (i.e. when given a list the output
    will always be the same, no matter the time of day, weather or else).
    
    \begin{ExampleIO}
    \egio{1 2 3 4 5}{3 4 5 2 1}
    \end{ExampleIO}
\end{task}

\begin{task}{Just Like In The Code!}{3}
    \In
    A list of space-separated words not containing any special symbols 
    (i.e. \href{https://en.wikipedia.org/wiki/Newline}{newline}s, 
    \href{https://en.wikipedia.org/wiki/Carriage_return}{carriage return}s, 
    \href{https://en.wikipedia.org/wiki/Bell_character}{bell}s), single or
    double quotes.
    \Out
    A list, formatted like it is written in python. See the examples below.
    
    \begin{ExampleIO}
    \egio{hi, oh dear world of sunshine}{["hi,", "oh", "dear", "world", "of", "sunshine"]}
    \end{ExampleIO}
\end{task}

\begin{task}{No-duplicated Merge}{4}
    \In
    Two lists of lenghts $n \geq 0$ and $m \geq 0$.
    \Out
    A list, consisting of all the elements from each list taken one by one
    without repeating duplicates. Follow the examples.

    \begin{ExampleIO}
    \egio{1 3 5 7\\2 4 6 8}{1 2 3 4 5 6 7 8}
    \egio{1 2 3\\1 2 3}{1 2 3}
    \egio{7 21 33 -4 60\\ 21 -4 57}{7 21 33 -4 60 57}
    \end{ExampleIO}
\end{task}

\begin{task}{Strong Neighbour}{3}
    \In
    A list of space-separated numbers $a_i$ of length $n \geq 1$.
    \Out
    A list of maximum elements from each \textbf{adjecent triplet} of numbers.

    \begin{ExampleIO}
    \egio{8 1 9 3 5 1 0 -8}{9 9 9 5 5 1}
    \end{ExampleIO}
\end{task}

\begin{task}{Transpose}{2}
    \In
    A square matrix of numbers of size $n \times n$.
    \Out
    A transpose of that matrix. The same matrix flipped across its main diagonal.

    \begin{ExampleIO}
    \egio{1 4 7\\2 5 8\\3 6 9}{1 2 3\\4 5 6\\7 8 9}
    \end{ExampleIO}
\end{task}

\begin{task}{Slice up}{2}
    \In
    First line is an integer $S$, the second line is a list $L$ of numbers containing $S$.
    \Out
    Two parts of the list, before and after $S$, both not including $S$.
    \begin{ExampleIO}
    \egio{0\\3 6 9 0 2 4 8}{3 6 9\\2 4 8}
    \egio{0\\0 88 77 66}{ \\88 77 66}
    \end{ExampleIO}
\end{task}

