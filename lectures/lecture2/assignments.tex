\begin{task}{Fair Square}{1}
\In
An integer $A$ such that $10 \leq A \leq 100$.
\Out
Using \s{from turtle import Turtle}'s methods like 
\s{forward} and \s{right} draw a square of length $A$.
\end{task}

\begin{task}{Fair Ngon}{2}
\In
Two integers, $A$ such that $10 \leq A \leq 100$ and $N$
such that $2 \leq N \leq 20$.
\Out
Using \s{Turtle} draw a regular polygon (an $N$-gon) with $N$ sides and side
length $5A$. Ensure that the turtle finishes in the same position as it started
in. The turtle shouldn't draw over itself at any point.
\Hint
Loops are your friend.
\end{task}

\begin{task}{Trigonometry BFF}{3}
\In
Two integers, $a$ and $b$.
\Out
Using \s{Turtle} draw a graph of the function $y = a * sin(\frac{\pi x}{10}) + b$.
From $0$ to $200$ and a graph of the function $y = b$. Print the final position
of the turtle.
\Hint
You can get $\sin$ and $\pi$ with \s{from math import pi, sin}, they are accurate
enough for this purpose.
\end{task}

\begin{task}{The Fair Ngon}{5}
\In
Two integers, $A$ such that $10 \leq A \leq 100$ and $N$
such that $2 \leq N \leq 20$.
\Out
Using \s{Turtle} draw a regular polygon (an $N$-gon) with $N$ sides and side
length $10A$. Ensure that the turtle finishes in the same position as it started
in. You are only allowed to control the turtle with \s{goto}.
\Hint
Trigonometry might help.
\end{task}

\begin{task}{Tick Space Tick Space Tick}{2}
\In
Two integers, $10 \leq L \leq 100$ and $1 \leq N \leq 15$.
\Out
Draw a horizontal dotted line of $N$ segments. 
The length of each segment should be $L$.
The space between two segments should also be $L$.
\Note
The turtle should start and end the drawing with a filled segment.
\Hint
Make use of \s{turtle.penup}, \s{turle.penup} and \s{turtle.isdown}.
\end{task}

\begin{task}{Fib}{2}
\In
An integer $n$, $n > 1$.
\Out
All terms from 0th to $n$th (inclusive) of the Fibonacci sequence.
The Fibonacci sequence is defined as follows.
$$F_0 = 0$$
$$F_1 = 1$$
$$F_n = F_{n - 1} + F_{n - 2}$$
\Note
You can solve this with both loops and recursion, maybe try it both ways?
You may use \href{https://oeis.org/}{this website} to check how correct your result is.
\begin{ExampleIO}
\egio{14}{0\\1\\1\\2\\3\\5\\8\\13\\21\\34\\55\\89\\144\\233\\377\\}
\end{ExampleIO}
\end{task}

\begin{task}{Blaise's Blessing}{3}
\In
An integer $N$, $N > 0$.
\Out
$N$ rows of the pascal triangle.
\Hint
The air is stale, open the \textit{window} please.

\begin{ExampleIO}
\egio{4}{1\\1 1\\1 2 1\\1 3 3 1\\1 4 6 4 1}
\end{ExampleIO}
\end{task}

\begin{task}{Poetic Crisis In Archaic Greece}{4}
Sappho has grown tired of poetry. She pleaded for Apollo to help her choose
the best so far. Being a busy god he refuses, yet he lets one of his nymphs
help the poor poet. Sappho layed out all of her poetry in a line and graded each
work with a mark $I \leq n \leq DCCC$. She didn't like some of her writings, so she
put a negative sign before those. After that she noticed that no two poems were
of equal ranking, they all recieved different grades, no two grades of different
poems were the same.
Nymph is a very joyful and not a very intellegint creature, she can only compare
two poems adjecent to eachother. They are also
inattentive enough to ignore Sappho explaination of the negative sign.

The nymph has started her work. She compares two works and loudly shouts
out the grade of the best poem and then moves on to the next pair. That continues
until there is only one poem left.

\In
An integer $1 < p < 100$, the amount of poems. 
Then, a sequence $N$ of space-separated integers $-800 \leq n \leq 800$, 
$n \neq 0$, $\forall a, b \in N : a \neq b$.
\Out
$p$ lines of space-separated integers representing all the sequence the nymph shouted out.
\Hint there is \s{abs(x)}, it is equivalent to $|x|$.

\begin{ExampleIO}
\egio{5\\-2 1 14 -17 5}{-2 1 14 -17 5\\-2 14 -17 5\\14 -17 -17\\-17 -17\\-17}
\end{ExampleIO}
\end{task}
