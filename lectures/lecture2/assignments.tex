
\begin{task}{Fair Square}{1}
\In
An integer $A$ such that $10 \leq A \leq 100$.
\Out
Using \s{from turtle import Turtle}'s methods like 
\s{forward} and \s{right} draw a square of length $A$.
\end{task}

\begin{task}{Fair Ngon}{2}
\In
Two integers, $A$ such that $10 \leq A \leq 100$ and $N$
such that $2 \leq N \leq 20$.
\Out
Using \s{Turtle} draw a regular polygon (an $N$-gon) with $N$ sides and side
length $A$. Ensure that the turtle finishes in the same position as it started
in. The turtle shouldn't draw over itself at any point.
\Hint
Loops are your friend.
\end{task}

\begin{task}{Trigonometry BFF}{3}
\In
Two integers, $a$ and $b$.
\Out
Using \s{Turtle} draw a graph of the function $y = a * sin(\frac{\pi x}{10}) + b$.
From $0$ to $20$ and a graph of the function $y = b$. Print the final position
of the turtle.
\Hint
You can get $\sin$ and $\pi$ with \s{from math import pi, sin}, they are accurate
enough for this purpose.
\end{task}

\begin{task}{The Fair Ngon}{5}
\In
Two integers, $A$ such that $10 \leq A \leq 100$ and $N$
such that $2 \leq N \leq 20$.
\Out
Using \s{Turtle} draw a regular polygon (an $N$-gon) with $N$ sides and side
length $A$. Ensure that the turtle finishes in the same position as it started
in. You are only allowed to control the turtle with \s{penup}, \s{pendown},
\s{goto}.
\Hint
Trigonometry might help.
\end{task}
