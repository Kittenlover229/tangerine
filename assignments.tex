\documentclass{article}
\usepackage{MnSymbol}
\usepackage{multido}
\usepackage{xfp}

\pagestyle{empty}

\newcommand{\s}[1]{\texttt{#1}}
\newcommand{\In}{\paragraph{Input:}}
\newcommand{\Out}{\paragraph{Output:}}
\newcommand{\Note}{\paragraph{Note:}}
\newcommand{\Hint}{\paragraph{Hint:}}

\newenvironment{task}[2]
    {\subsection{#1 
        \multido{\i=1+1}{#2}{$\filledstar$}
        \multido{\i=1+1}{\fpeval{5-#2}}{$\smallstar$}
        }
        \begin{minipage}{\linewidth}}
    {\end{minipage}}


\begin{document}

\setcounter{section}{-1}
\section{Prelude}

\begin{flushleft}
\textbf{How are the tasks rated?}
\end{flushleft}
\begin{flushright}
I want to say randomly, but they aren't. This is my own estimate of how I think
my students will perform based on their level and of the quality of the lecture
I've given previously. 5 stars are the exercises that will take them a
significant amount of time and may requiure external knowledge/research.
The one 1 stars easy and shouldn't take much effort at all. All the others lie 
somewhere inbetween on this range.
\end{flushright}

\begin{flushleft}
\textbf{I'm stuck! What next?}
\end{flushleft}
\begin{flushright}
Do let me know if some exercise made you seriously stumble. Take a break and
go for a walk, maybe try solve it tomorrow, perhaps you are tired. If you can't
crach that exercise to matter what be sure to inform me at the start of our
next lesson, we'll go through it.
\end{flushright}

\begin{flushleft}
\textbf{Why is the document so exquisitely formatted?}
\end{flushleft}
\begin{flushright}
I like it more that way. I am also practising \LaTeX\ in general,
it's really lovely.
\end{flushright}

\tableofcontents
    

\section{Simple Stuff}
Prerequisites: control flow (branching, iteration), IO, arithmetic, atomic types.
\subsection{The Good Old Days $\star$}
\In
An integer $4$.
\Out
The word \s{"Elephant"}.

\subsection{Equation of a Line $\star$}
\In
Two integers $k$ and $b$, $k \neq 0$.
\Out
Such value $x$, that it satisfies the equation $kx+b=0$. 

\subsection{Wait, what? $\star\star$}
\In
Two integers $a$ and $b$.
\Out
The product of $a$ and $b$.
\Note
You may not use the multiplication operation.

\subsection{Late'o'clock $\star\star$}
\In An integer $0 \leq h < 24$. Hours on a clock.
\Note Convert the given time $h$ to the 12-hour clock format.
\Out First the time $h$ in 12-hour clock format, then \s{"am"} or \s{"pm"}
depending on the time.

\subsection{Quadratic Equation $\star\star$}
\In
Three integers $a$, $b$ and $c$.
\Out
Find all values of $x$, such that $ax^2 + bx + c=0$. 
\Note
If there are no possible values of $x$ output \texttt{"NaN"} (not a number). 
The values should not be repeated.

\subsection{Qubic Equation $\star\star\star$}
\In
Four integers $a$, $b$, $c$ and $d$.
\Out
Find all values of $x$, such that $ax^3 + bx^2 + cx + d = 0$. 
\Note
If there are no possible values of $x$ output \texttt{"NaN"} (not a number). 
The values should not be repeated.
\Hint
use Cardano's formula.

\subsection{Euclid Approves $\star$}
\In
Two integers $a$ and $b$, sides of a right angled triangle.
\Out
The hypotenuse $c$ of the aforementioned triangle.

\subsection{Euclid Disapproves $\star\star$}
\In
Two integers $a$ and $b$, sides of a right angled triangle and an integer angle
$\theta$ (given in degrees) between them.
\Out
The third side of the triangle.
\Hint
You may use \s{import math} to get some functions you might want.

\subsection{Everyone but Euclid Approves $\star\star\star$}
\In
An integer $n$ the amount of following lines, $3 \leq n \leq 100$. 
Each following line $i$ contains a number $-100 \leq a_i \leq 100$, a 
component of the vector $\hat{v} = \{a_1, a_2, \dots, a_n\}$.
\Out
The length of a vector $||\hat{v}||$.

\subsection{Minmaxed $\star$}
\In
Two integers, $a$ and $b$.
\Out
Two integers, first the largest of them two, next the smallest.

\subsection{TreE $\star\star\star$}
\In
An integer $h$, the height of the christmass tree.
\Out
A christmas tree with total height $h + 1$, $1$ being the trunk of said
tree and $h$ all the result of it.
\begin{verbatim}
e
a a
e e e 
a a a a
a
\end{verbatim}

\subsection{Sigma for Sum $\star\star$}
\In
An integer $a$ such that $1 \leq a \leq 10^{10^{10}}$.
\Out
The sum all the integers $1 + 2 + \dots + a$.
\Hint
Loop isn't the only way to go.

\subsection{Factor!al $\star\star\star$}
\In
An integer $a$ such that $1 \leq b \leq 10^{5}$.
\Out
The product all the integers $1 \times 2 \times \cdots \times b$.
\Hint
Lookup the arguments for \s{range} in the official Python3.x documentation.

\subsection{Minmaxed 2: The Sequel $\star\star\star$}
\In
Two integers, $a$ and $b$.
\Out
Two integers, first the largest of them two, next the smallest.
\Note
You may  only use \s{min()} or \s{max()}, not both. You may not use branching.


\section{Functions and Drawing}
Prerequisites: \s{turtle} module, lists, functions, recursion, induction.
\begin{task}{Fair Square}{1}
\In
An integer $A$ such that $10 \leq A \leq 100$.
\Out
Using \s{from turtle import Turtle}'s methods like 
\s{forward} and \s{right} draw a square of length $A$.
\end{task}

\begin{task}{Fair Ngon}{2}
\In
Two integers, $A$ such that $10 \leq A \leq 100$ and $N$
such that $2 \leq N \leq 20$.
\Out
Using \s{Turtle} draw a regular polygon (an $N$-gon) with $N$ sides and side
length $5A$. Ensure that the turtle finishes in the same position as it started
in. The turtle shouldn't draw over itself at any point.
\Hint
Loops are your friend.
\end{task}

\begin{task}{Trigonometry BFF}{3}
\In
Two integers, $a$ and $b$.
\Out
Using \s{Turtle} draw a graph of the function $y = a * sin(\frac{\pi x}{10}) + b$.
From $0$ to $200$ and a graph of the function $y = b$. Print the final position
of the turtle.
\Hint
You can get $\sin$ and $\pi$ with \s{from math import pi, sin}, they are accurate
enough for this purpose.
\end{task}

\begin{task}{The Fair Ngon}{5}
\In
Two integers, $A$ such that $10 \leq A \leq 100$ and $N$
such that $2 \leq N \leq 20$.
\Out
Using \s{Turtle} draw a regular polygon (an $N$-gon) with $N$ sides and side
length $10A$. Ensure that the turtle finishes in the same position as it started
in. You are only allowed to control the turtle with \s{goto}.
\Hint
Trigonometry might help.
\end{task}

\begin{task}{Tick Space Tick Space Tick}{2}
\In
Two integers, $10 \leq L \leq 100$ and $1 \leq N \leq 15$.
\Out
Draw a horizontal dotted line of $N$ segments. 
The length of each segment should be $L$.
The space between two segments should also be $L$.
\Note
The turtle should start and end the drawing with a filled segment.
\Hint
Make use of \s{turtle.penup}, \s{turle.penup} and \s{turtle.isdown}.
\end{task}

\begin{task}{Fib}{2}
\In
An integer $n$, $n > 1$.
\Out
All terms from 0th to $n$th (inclusive) of the Fibonacci sequence.
The Fibonacci sequence is defined as follows.
$$F_0 = 0$$
$$F_1 = 1$$
$$F_n = F_{n - 1} + F_{n - 2}$$
\Note
You can solve this with both loops and recursion, maybe try it both ways?
You may use \href{https://oeis.org/}{this website} to check how correct your result is.
\begin{ExampleIO}
\egio{14}{0\\1\\1\\2\\3\\5\\8\\13\\21\\34\\55\\89\\144\\233\\377\\}
\end{ExampleIO}
\end{task}

\begin{task}{Blaise's Blessing}{3}
\In
An integer $N$, $N > 0$.
\Out
$N$ rows of the pascal triangle.
\Hint
The air is stale, open the \textit{window} please.

\begin{ExampleIO}
\egio{4}{1\\1 1\\1 2 1\\1 3 3 1\\1 4 6 4 1}
\end{ExampleIO}
\end{task}

\begin{task}{Poetic Crisis In Archaic Greece}{4}
Sappho has grown tired of poetry. She pleaded for Apollo to help her choose
the best so far. Being a busy god he refuses, yet he lets one of his nymphs
help the poor poet. Sappho layed out all of her poetry in a line and graded each
work with a mark $I \leq n \leq DCCC$. She didn't like some of her writings, so she
put a negative sign before those. After that she noticed that no two poems were
of equal ranking, they all recieved different grades, no two grades of different
poems were the same.
Nymph is a very joyful and not a very intellegint creature, she can only compare
two poems adjecent to eachother. They are also
inattentive enough to ignore Sappho explaination of the negative sign.

The nymph has started her work. She compares two works and loudly shouts
out the grade of the best poem and then moves on to the next pair. That continues
until there is only one poem left.

\In
An integer $1 < p < 100$, the amount of poems. 
Then, a sequence $N$ of space-separated integers $-800 \leq n \leq 800$, 
$n \neq 0$, $\forall a, b \in N : a \neq b$.
\Out
$p$ lines of space-separated integers representing all the sequence the nymph shouted out.
\Hint there is \s{abs(x)}, it is equivalent to $|x|$.

\begin{ExampleIO}
\egio{5\\-2 1 14 -17 5}{-2 1 14 -17 5\\-2 14 -17 5\\14 -17 -17\\-17 -17\\-17}
\end{ExampleIO}
\end{task}


\clearpage
\doclicenseThis

\end{document}
